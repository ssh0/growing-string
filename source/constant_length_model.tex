% <- main.tex
\subsubsection{概要}

自然長を増大させるモデルにおいて,線分の交差判定を行う際に困難が生じた。

成長させているのは自然長であるため,一般に剛体のシミュレーションとして行うときのような剛体球ポテンシャルは有効でなく,したがって,実際に点を動かす前にその次のステップで交差するかどうかを判定した上で,もし交差しそうなら交差しないように次のステップで移動する点の位置を変更する作業が必要だった。
しかしながら, (もう少し頑張れるかもしれないが)この試みはうまくいかなかったので,
今度は発想を変えて自然長,バネ定数$K$は一定に保ったまま,ある時間間隔ごとに
ランダムに点を1つずつ追加していくことによって線素群の成長を記述することを試みる。
また,自己回避的な挙動を示して交差を防ぐために,各点ごとに斥力ポテンシャルを設け,
交差が起こらないようにする。

\subsubsection{結果}

実際にシミュレーションとして作動させてみたが,うまいパラメータの設定ができていないのか,
期待通りに動かなかった。

また,質点間距離はある程度抑えられているとはいえ,互いの距離を一定に保ったまま動的に動かすことは難しい。

(要確認 & 再挑戦)
